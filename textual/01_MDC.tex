La educación superior es un proceso permanente que posibilita el fomento de las potencialidades del ser humano de una manera integral y tiene por objeto el pleno desarrollo de los alumnos y su formación académica o profesional (art. 1°, Ley 30 de 1992). La educación tiene la misión de permitir a todas las personas hacer fructificar todos sus talentos y capacidades de creación, lo que se resume en el término de formación integral, entendida como el desarrollo tanto de competencias transversales y disciplinares; de aptitudes culturales, artísticas, deportivas; y de cualidades personales para relacionarse con los demás.

En el término de Gardner (1995) , el desarrollo de dichas competencias no son el resultado de procesos innatos ni mucho menos predeterminados; es decir, no se nace destinado para desarrollar una competencia. Existen múltiples inteligencias que permiten que cada persona, con su propia inteligencia, pueda desarrollar competencias a partir de la exigencia y estímulos del entorno.  En este orden de ideas, centrarse en crear dicho entorno para la formación de competencias que permitan el desarrollo integral del ser humano, convirtiendo el aprendizaje, y no a la enseñanza, en el corazón de la formación. Tratándose del Doctorado en Economía, dicho proceso de formación se fundamenta en la formación de investigadores de excelencia, creativos, críticos, analíticos y autónomos capaces de expandir la frontera del conocimiento en economía en general y en el campo de especialización elegido en particular mediante la realización de investigaciones de alta relevancia para la solución de problemas contemporáneos a nivel nacional e internacional. 

Bajo este enfoque, el objetivo del programa de Doctorado en Economía es el siguiente: 

\begin{tcolorbox}[colback=blue!5!white,colframe=blue!75!black,title=Objetivo del programa]
  El Doctorado en Economía de la Universidad EAFIT tiene como objetivo principal formar investigadores de excelencia, creativos,  integrales, críticos y autónomos; competentes a nivel nacional e internacional; capaces de expandir la frontera del conocimiento en economía en general y en su campo de especialización en particular mediante la realización de investigaciones de alta calidad con implicaciones prácticas relevantes para el desarrollo empresarial, gubernamental y académico del país y del mundo.
\end{tcolorbox}

Con el fin de cumplir este objetivo, el Doctorado en Economía de la Universidad EAFIT ha adoptado un enfoque holístico del desarrollo de investigadores el cual busca contribuir al desarrollo de competencias desde lo disciplinar, personal, profesional y de compromiso con el medio. Este enfoque es el resultado de una conceptualización basado en diferentes marcos los cuales fueron revisados a la luz del plan estratégico institucional
\footnote{Entre los principales referentes revisados se encuentran: 
\begin{itemize}
    \item La adaptación del proyecto Tunning para Latinoamérica, el cual provee 27 competencias genéricas que responden a los lineamientos del acuerdo de Bolonia y que además parten de las realidades de Latinoamérica. 
    \item El Marco para Desarrollo del Investigador (MDI) del proyecto Vitae, una organización dedicada a aprovechar el potencial de los investigadores mediante la transformación de su desarrollo profesional y de carrera. El MDI fue creado a partir de una extensiva recolección de datos a través de entrevistas con investigadores alrededor mundo, con el fin de identificar las características profesionales y personales que tienen un investigador de excelencia. A partir de allí, ofrece una clasificación de las competencias en cuatro dimensiones que agrupan dichas características, creando un enfoque holístico sobre la formación de investigadores
    \item Los lineamientos para la formación por competencias en educación superior creado por el Ministerio de Educación en el 2015, el cual propone 58 competencias dividas en tres dimensiones: (i) competencias abstractas del pensamiento (24 competencias); (ii) conocimiento y competencias prácticas (31 competencias); y, (iii) dinamizadores para el desarrollo de competencias abstractas (3 competencias).
    \item Diseño por competencias de referentes internacionales de otros programas doctorales en: Latinoamérica, Europa y Norte América .
\end{itemize}}. Como resultado de este proceso, se creó el Marco de Desarrollo de Competencias (MDC) del Doctorado en Economía de la Universidad EAFIT, el cual está anclado a la impronta de EAFIT y por tanto constituye la marca de identidad del programa y la institución.  

\begin{figure}[H]
\caption{Marco de Desarrollo de Competencias (MDC) \label{map_result_2} }
\begin{center}
\includegraphics[width=\textwidth,keepaspectratio]{img/MDC_Basico.png}
\end{center}
\end{figure}

El marco de desarrollo de competencias del Doctorado en Economía (MDC) brinda una visión holística e integral sobre la formación de los Doctores en Economía de la Universidad EAFIT, permitiendo crear una identidad que cumple el objetivo del programa. Las competencias que se espera desarrollar en cada dimensión del MDC son las siguientes: 

\begin{itemize}
    \item \textbf{Conocimientos y habilidades intelectuales: } Esta dimensión hace referencia al conocimiento, habilidades intelectuales y técnicas sobre las bases conceptuales y tendencias temáticas de las ciencias económicas. Aquí se hace un especial énfasis, en el desarrollo de competencias para el desempeño del estudiante en su áreas de investigación de cada estudiante. Así, se espera que los egresados, puedan desarrollar la siguiente competencia:
    
    \begin{tcolorbox}[colback=red!5!white,colframe=red!75!black]
  Desarrolla de manera crítica conocimiento innovador y pertinente en su área de investigación, a partir de una comprensión profunda y holística de los fundamentos, tendencias temáticas y metodológicas de las ciencias económicas.
\end{tcolorbox}

\item \textbf{Efectividad personal:} Para esta dimensión, se busca crear un entorno para fomentar que los estudiantes del doctorado pueden tener una amplia comprensión de las implicaciones, acciones y planes para el desarrollo de sus carrera profesional no solo dentro del programa doctoral, sino también del aprendizaje a lo largo de la vida desde su quehacer profesional como investigadores, docentes, posiciones gerenciales, entre otros áreas de desempeño. Así, la competencia que se busca desarrollar es la siguiente:
    
        \begin{tcolorbox}[colback=red!5!white,colframe=red!75!black]
Evidencia un compromiso para la consolidación de su ejercicio profesional basado en la alta calidad, equidad, diversidad, ética y el equilibrio entre la vida laboral y el bienestar en la vida profesional.
\end{tcolorbox}

    \item \textbf{Gobernanza de la investigación:} Esta dimensión hace referencia a una de las principales competencias de un investigador: la capacidad de planear, dirigir y ejecutar proyectos. Así, la formación del doctorado buscará crear los conocimientos de los estándares, requisitos y profesionalismo para investigar en las ciencias Económicas. Esto se resume en la siguiente competencia:
    
        \begin{tcolorbox}[colback=red!5!white,colframe=red!75!black]
 Gestiona proyectos de investigación innovadores de manera efectiva y sistemática, que contribuyen al avance de la ciencia económica y el mejoramiento de las condiciones de vida de la sociedad..
\end{tcolorbox}


 \item \textbf{Compromiso e influencia en el medio:} En la última dimensión se resalta la importancia de los conocimientos y habilidades para asegurar que los resultados de investigación de los investigadores, puedan tener un impacto en el medio que se desempeñan. Esto se resume en la siguiente competencia:
    
        \begin{tcolorbox}[colback=red!5!white,colframe=red!75!black]
Impacta positivamente en el medio a través de actividades de investigación, alta gerencia, consultoría y/o docencia usando una comunicación efectiva hacia la comunidad académica, hacedores de política y líderes empresariales.
\end{tcolorbox}
    
\end{itemize}

El  MDC permite no solo guiar los aspectos curriculares del programa, sino también clarifica los perfiles de ingreso y egreso del programa. En términos conceptuales, siguiendo el modelo de desarrollo propuesto por Dreyfus , se espera que en el momento del ingreso los aspirantes al programa estén en un nivel de principiantes avanzado en las competencias del MDC, lo cual garantiza que conoce y entiende los principios básicos de las implicaciones personales y profesionales de la investigación en Economía. Y, una vez cumpla con los requisitos académicos del programa, se espera que los egresados tengan los elementos suficiente para crear una experticia en las dimensiones del MDC, lo cual garantizará que pueden evaluar y crear conocimiento de frontera que lo convertirá en un investigador de excelencia. 

\begin{figure}[H]
\caption{Perfil de ingreso y egreso basado en el Marco de desarrollo de competencias para los estudiantes del Doctorado en Economía de la Universidad EAFIT \label{mdc_niveles} }
\begin{center}
\includegraphics[width=\textwidth,keepaspectratio]{img/MDC_Escala.png}
\end{center}
\end{figure}

En este orden de ideas, el perfil de ingreso de los aspirantes al programa se resume de la siguiente manera:

\begin{tcolorbox}[colback=blue!5!white,colframe=blue!75!black,title=Perfil de Ingreso]
  Los aspirantes al Doctorado en Economía de la Universidad EAFIT deben demostrar un amplio compromiso y disponibilidad para formarse como investigadores de alto nivel en las ciencias Económicas. El candidato debe tener excelentes conocimientos y habilidades intelectuales en términos analíticos y cuantitativos. Además, debe demostrar cualidades personales que permitan el desarrolla de las competencias necesarias para crear y dirigir investigaciones originales en un campo de especialización de su escogencia.
\end{tcolorbox}

El grado de Doctor en Economía otorgado por la Universidad EAFIT confiere un reconocimiento único a quien se le otorga de que cumple con la cualificación suficiente y general de un economista y que tiene las competencias necesarias para realizar investigaciones de alto nivel en un campo de especialización en economía que contribuyan significativamente al avance del conocimiento en su área de experticia. De esta manera, una vez se cumpla con los diferentes resultados de aprendizaje del programa doctoral, se espera que el egresado del doctorado en Economía adquiere los conocimientos y desarrolla las competencias necesarias para cumplir con el siguiente perfil ocupacional:

\begin{tcolorbox}[colback=blue!5!white,colframe=blue!75!black,title=Perfil Ocupacional]
  Los futuros Doctores en Economía de EAFIT podrán desempeñarse como docentes e investigadores en universidades nacionales y extranjeras, en centros de investigaciones, ocupar posiciones destacadas en el sector público, en empresas del sector privado, en entidades multilaterales o desempeñarse como consultores expertos El perfil de los futuros doctores en economía de la Universidad EAFIT reflejará las competencias propias a desarrollar del marco de competencias del Doctorado en Economía. 

\end{tcolorbox}

Esto lo garantiza el siguiente perfil del egresado:

\begin{tcolorbox}[breakable,colback=blue!5!white,colframe=blue!75!black,title=Perfil de egreso]
  Los Doctores de Economía de la Universidad EAFIT reflejarán un alto nivel de desarrollo de las  competencias propias del marco de competencias del programa enmarcados en los valores institucionales de tolerancia, audacia, excelencia, responsabilidad e integridad . Así, los egresados se distinguirán por ser tener capacidades en las siguientes dimensiones:
  
  \begin{itemize}
      \item \textbf{Conocimiento y habilidades intelectuales: }
      \begin{itemize}
          \item Demostrar un conocimiento profundo de la bibliografía y una comprensión exhaustiva de los métodos y técnicas aplicables a su propia investigación.
          \item Tener excelentes habilidades cognitivas que demuestren un entendimiento profundo, propio de un experto, respecto a los conocimientos teóricos en economía en general y de su campo de conocimiento en particular.
          \item Descubrir, interpretar y comunicar nuevos conocimientos mediante una investigación original de calidad publicable que satisfaga la revisión por pares.
      \end{itemize}
      
            \item \textbf{Efectividad personal: }
      \begin{itemize}
          \item Aplicar una gama importante de conocimientos avanzados y especializados y ser capaz de actuar de forma autónoma en la planificación y ejecución de la investigación.
        \item Practicar un enfoque proactivo, autocrítico y autorreflexivo basado en la investigación y desarrollar relaciones profesionales con otros cuando sea apropiado.
        \item Demostrar liderazgo y originalidad a la hora de abordar y trabajar en grupos diversos, mediante la comunicación y el trabajo eficaz con los demás.
      \end{itemize}

  
  
    \item \textbf{Gobernanza de la investigación: }
      \begin{itemize}
          \item Presentar y defender resultados de investigación originales que amplíen la vanguardia de una disciplina o área relevante de la práctica profesional.
          \item Evaluar de forma crítica y creativa los temas de actualidad, la investigación y los estudios avanzados en la disciplina.
          \item Gestionar cuestiones éticas y profesionales complejas y emitir juicios fundados sobre los códigos y las prácticas éticas.

      \end{itemize}
      
          \item \textbf{Compromiso e influencia en el medio: }
      \begin{itemize}
          \item Trabajar en colaboración con todas las partes interesadas para crear, desarrollar e intercambiar conocimientos de investigación que influyan y beneficien a la sociedad y/o organizaciones donde se desempeña.
          \item Capacidad de comunicar su trabajo de forma efectiva para diferentes audiencias, bajo los más altos estándares, tanto en forma escrita como oral.
          \item Ser capaces de referirse con experticia y tomar posiciones respecto a la situación general de las organizaciones, la región, el país, y el mundo; y ser capaces de emitir juicios propios de un experto en su área de especialización respecto a situaciones específicas. 
      \end{itemize}
  \end{itemize}

\end{tcolorbox}

A continuación se presenta la materialización del MDC en términos de los resultados de aprendizaje y su articulación al plan de estudios.