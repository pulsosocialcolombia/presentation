El Examen comprensivo de profundización es la última etapa del ciclo de profundización y abre el camino hacia el inicio de la investigación doctoral. En sentido, este examen tiene por objetivo evaluar el dominio del estudiante respecto a los contenidos fundamentales de su línea de profundización, además de las habilidades metodológicas y profesionales necesarias para su formación como investigador.

El Examen Compresivo de Profundización tiene por objetivo evaluar el dominio del estudiante respecto a los contenidos fundamentales de su línea de profundización, además de las habilidades metodológicas y profesionales necesarias para su formación como investigador. En particular, el examen busca evaluar las siguientes competencias:

\begin{itemize}
    \item Capacidad de expresión oral.
    \item Búsqueda e identificación de la literatura relevante.
    \item Identificación de frontera de conocimiento.
    \item Articulación con la investigación que realiza el estudiante.
\end{itemize}

El curso se enmarca en el Marco de Desarrollo de Competencias -MDC- de la Escuela de Economía y Finanzas, contribuyendo a la consolidación de las siguientes áreas de dominio y descriptores:

\begin{figure}[H]
\caption{Articulación del Examen de Profundización al  MDC \label{hito1_mdc} }
\begin{center}
\includegraphics[width=\textwidth,keepaspectratio]{img/mdc_hito2.png}
\end{center}
\end{figure}

De esta manera, al final el curso lo estudiantes habrán fortalecido las siguientes competencias:
 
\section{Resultados de aprendizaje}

En el área de dominio Conocimiento y habilidades intelectuales, los resultados de aprendizaje son los siguientes:

\begin{itemize}

\item 	Demuestra madurez en la conceptualización y formalización de conceptos matemáticos (A1.3)
\item 	Domina los fundamentación teórica referente a la teoría del productor y la teoría del consumidor. (A1.5)
\item 	Comprende los supuestos fundamentales de la teoría de la utilidad esperada y sus aplicaciones (A1.8)
\item 	Demuestra conocimiento sobre los fundamentos macroeconómicos de una economía que se encuentra en estado estable. (A1.9)
\item 	Demuestra un conocimiento a profundidad del estimador de mínimos cuadrados ordinarios, mínimos cuadrados generalizados, variable instrumental, máxima verosimilitud, método de los momentos y método generalizado de los momentos. (A1.13)
\item 	Presenta el desarrollo formal de conceptos fundamentales para modelos no-lineales de corte transversal y datos de panel. (A1.15)
\item 	Identifica los elementos empíricos característicos de los ciclos económicos (A2.11)
\item 	Identifica los elementos característicos de los ciclos económicos en un un contexto de pequeñas economías abiertas . (A2.13)

\end{itemize}


\section{Dinámica del Examen}

El Examen Compresivo de profundización está compuesto por cinco momentos:

\begin{tikzpicture}
\draw[thick, -Triangle] (0,0) -- (\ImageWidth,0) node[font=\scriptsize,below left=3pt and -8pt]{};

% draw vertical lines
\foreach \x in {0,4,8}
\draw (\x cm,4pt) -- (\x cm,-4pt);

\foreach \x/\descr in {0/Inicio,4/Entrega,8/Presentación}
\node[font=\scriptsize, text height=1.75ex,
text depth=.5ex] at (\x,-.3) {$\descr$};


% braces
\draw [thick ,decorate,decoration={brace,amplitude=5pt}] (4,-0.7)  -- +(-4,0) 
       node [black,midway,below=4pt, font=\scriptsize] {Preparación Estudiante};
\draw [thick ,decorate,decoration={brace,amplitude=5pt}] (4,-0.7)  -- +(-4,0) 
       node [black,midway,above=4pt, font=\scriptsize] {\textbf{72 Horas}};
\draw [thick,decorate,decoration={brace,amplitude=5pt}] (8,-.9) -- +(-4,0)
       node [black,midway,font=\scriptsize, below=4pt] {Revisión Comité};
\draw [thick,decorate,decoration={brace,amplitude=5pt}] (8,-.9) -- +(-4,0)
       node [black,midway,font=\scriptsize, above=4pt] {\textbf{48 horas}};
\end{tikzpicture}

\begin{itemize}
    \item \textbf{Inicio:} Una vez el comité evaluador defina el \emph{tema o pregunta de investigación}, el estudiante es notificado vía e-mail sobre el tema, el inicio de las 72 horas, la programación de la presentación \emph{hora y fecha y hora límite para del envió}. El estudiante tendrá solo \emph{una (1) oportunidad} para hacer preguntas aclaratorias a través de los medios oficiales del doctorado. No se permite un contacto con el comité evaluador durante el periodo del examen.
    \item \textbf{Preparación por parte del estudiante:} Durante este periodo se espera que el estudiante realice la escritura del informe siguiendo los parámetros que se mencionan a continuación. 
    \item \textbf{Entrega del informe escrito:} El estudiante deberá enviar al correo electrónico del doctorado (\href{mailto:phdeconomia@eafit.edu.co}{phdeconomia@eafit.edu.co}) el informe en formato (PDF). Este envío debe hacerse antes de la hora final, en caso de enviarla después de la hora establecida el examen quedará automáticamente como \emph{reprobado}.
    \item \textbf{Presentación:} Después de la revisión del informe por parte del comité evaluador, el estudiante tendrá que realizar la presentación oral del informe.
\end{itemize}

\section{Características de la pregunta de investigación}

El comité de evaluación tendrá la responsabilidad de asignar una pregunta o tema de investigación a fin al área del conocimiento de la tesis doctoral del estudiante
\begin{itemize}
    \item \textbf{Modalidad pregunta de investigación:} El comité crea una pregunta de investigación específica en el área de trabajo del estudiante, la cual puede deber ser resuelta a través del desarrollo de un modelo teórico o a través del uso de métodos estadísticos/econométricos con datos reales.
    \item \textbf{Modalidad área de conocimiento:} El comité designa un área de conocimiento a partir de la cual el estudiante deberá realizar una búsqueda para identificar las principales preguntas de investigación de dicha área y sus respectivos abordes teóricos y metodológicos en la literatura económica.
\end{itemize}

En cualquiera de las dos modalidades, el estudiante debe:

\begin{itemize}
    \item Realizar una revisión de literatura exhaustiva y actualizada, evitando el uso de referencias grises (no publicadas y/o publicadas en revistas sin validación académica) u otras fuentes de información no académicas.
    \item La estructura del trabajo debe seguir una cohesión y coherencia clara, que garantice un lenguaje académico claro.
    \item En caso de realizar el uso de información o datos estadísticos, el estudiante debe garantizar el uso adecuado de los datos, citas y demás aspectos de formato propios del lenguaje académico.
\end{itemize}

\subsection{Parámetros del trabajo escrito}

El informe inscrito tendrá las siguientes características:
\begin{itemize}
    \item 15 páginas (máximo)
    \item La estructura del informe deberá tener, como mínimo las siguientes secciones:
    \begin{itemize}
        \item Motivación
        \item Marco Conceptual
        \item Respuesta a la pregunta de investigación (\emph{modalidad de pregunta de investigación})
        \item Principales preguntas en el área (\emph{modalidad área de conocimiento})
    \end{itemize}
    \item El trabajo deberá presentarse preferiblemente en Inglés. En caso contrario, el estudiante debe informar a la coordinación.
    \item El estudiante podrá realizar el trabajo en \LaTeX, Word u otro procesador de texto, siempre y cuando garantice los parámetros de calidad y presentación exigidos.
\end{itemize}

En caso que el comité de evaluación considere que el informe debe tener una estructura diferente, este será informado al estudiante. 

\subsection{Parámetros de la presentación}

Durante la presentación el estudiante tendrá un tiempo 20 minutos de presentación (deben usar el formato de presentación del doctorado). Posteriormente, el comité evaluador tendrá 15 minutos de preguntas. Luego el estudiante es retirado de la sala, para que el comité defina si el estudiante reprobó o aprobó el examen.

\section{Conformación y responsabilidad del Comité Evaluador}

La evaluación del Examen Compresivo III está a cargo de un grupo evaluador conformado por dos (2) profesores: (i) el asesor del estudiante; y, (ii) un (1) profesor afín al área metodológica y/o temática de la Escuela de Economía y Finanzas, el cual es seleccionado entre la Dirección del Doctorado y el asesor.
El Comité Evaluador tendrá la responsabilidad de:

\begin{itemize}
    \item Asignar el tema o pregunta de investigación según la modalidad descritas anteriormente.
    \item Garantizar la asistencia (presencial o virtual) a la presentación oral del estudiante.
    \item Proveer a la Dirección del Doctorado la calificación final de Aprobado o Reaprobado.
\end{itemize}



