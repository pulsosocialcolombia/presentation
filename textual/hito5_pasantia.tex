La defensa de tesis constituye el hito más importante del Doctorado, porque es en este momento donde se materiliza el desarrollo de todas las competencias del Marco de Desarrollo de Competencias del Doctorado en Economía. En este sentido, se espera que los estudiantes logren demostrar que han desarrollado todas las competencias del perfil de egreso del programa.

La defensa se enmarca en el Marco de Desarrollo de Competencias -MDC- de la Escuela de Economía y Finanzas, contribuyendo a la consolidación de las siguientes áreas de dominio y descriptores:

La pasantía se enmarca en el Marco de Desarrollo de Competencias -MDC- de la Escuela de Economía y Finanzas, contribuyendo a la consolidación de las siguientes áreas de dominio y descriptores:

\begin{figure}[H]
\caption{Articulación de la pasantía internacional al MDC\label{hito1_mdc} }
\begin{center}
\includegraphics[width=\textwidth,keepaspectratio]{img/mdc_hito5.png}
\end{center}
\end{figure}

\section{Resultados de Aprendizaje}

En el área de dominio Efectividad Personal, los resultados de aprendizaje son los siguientes:
\begin{itemize}
\item 	Lidera y participa en redes, dentro y fuera de la academia, para aumentar la visibilidad y el impacto de su investigación. (B3.3)
\end{itemize}
En el área de dominio Compromiso e Influencia en el Medio, los resultados de aprendizaje son los siguientes:
\begin{itemize}
\item 	Presenta su trabajo con cuidado y rigor científico usando un lenguaje adecuadad y un buen manejo del tiempo en los diferentes escenarios académicos (D2.1)
\item 	Interactúa con criterio y actitudes de respeto ante la diversidad articulando lo local y lo global (D3.3)
\item 	Su actitud es la de un ciudadano global inmerso en la economía internacional y responsable por realizar contribuciones que expandan la frontera del conocimiento en economía. (D3.4)
\item 	Tiene la habilidad de llevar el conocimiento de la investigación hacia el proceso de formulación de política a través de una variedad de mecanismos. (D3.5)
\item 	Promueve la transformación de contextos sociales y la construcción de civilidad y humanidad desde la perspectiva de respecto y valoración de la diversidad (D3.6)
\end{itemize}
