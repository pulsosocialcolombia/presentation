El seminario tiene por objetivo propiciar un espacio académico de alto nivel donde los candidatos a Doctor en Economía pueden recibir retroalimentación por parte de profesores y pares académicos, con el fin de brindar insumos para la consolidación de sus tesis. Adicionalmente, se busca que los candidatos consoliden sus habilidades de comunicación en ambientes académicos.

El seminario está dirigido aquellos estudiantes que tienen registrado la asignatura de Seminario II y III, los cuales deben presentar en este seminario como requisito para aprobar dicha asignatura. En casos excepcionales, se permitirá que otros estudiantes del doctorado presenten bajo la solicitud del asesor.

La dinámica del Seminario Doctoral seguirá la estructura de un seminario estándar en Economía. Cada uno de los presentadores deberá:
\begin{itemize}
    \item Enviar el documento completo quince (15) días antes del seminario para ser entregado a los lectores. 
    \begin{itemize}
        \item No existe ninguna restricción sobre el capítulo de tesis a presentar. Idealmente, se espera que los estudiantes puedan presentar su segundo artículo (artículo de tercer año). No obstante, esta decisión la debe tomar con su respectivo asesor.
    \end{itemize}
    \item El presentador debe preparar una presentación de 45 minutos de su trabajo de grado, siguiendo los contenidos básicos de un seminario académico. 
    \begin{itemize}
        \item  Ver la sección de recursos para algunas recomendaciones generales para la construcción de la propuesta.
    \end{itemize}

    \item Cada uno de los presentadores se les asignará dos lectores (1 profesor y 1 par del doctorado), para realizar una lectura crítica del trabajo. Estos preparan una intervención de 5 minutos discutiendo los principales aportes y posibles puntos a mejorar del artículo. 
\end{itemize}

\section{Estructura de la presentación}

Cada candidato tendrá de 50 minutos, la cual estará distribuida de la siguiente manera:
\begin{itemize}
    \item 30 minutos presentación
    \item 5 minutos para los comentarios del lector 1 – Profesor
    \item 5 minutos para los comentarios del lector 2 – Par académico
    \item 10 minutos para discusión y preguntas del público
\end{itemize}

\section{Recursos para la preparación del Seminario}

\begin{itemize}
    \item Osborne. Undated. "How to create good-looking figures for a LaTeX document." - Figures are often very useful to present your theory in an accessible way. This document explains how to use PSTricks to design a figure whose appearance is consistent with Tex documents.
    \item Schwabish, Jonathan A. 2014. “An Economist’s Guide to Visualizing Data.” Journal of Economic Perspectives 28(1): 209–34. --- Economists are typically terrible at visualising data. This paper corrects 10 graphs in the actual published research, to elucidate us on how to effectively convey information through visual images.
    \item David Levine (undated) "David Levine's Cheap Advice for Presenting Results": This is very useful mainly for empirical researchers.
    \item Davis (undated) "What Makes for a Successful Paper and Seminar?": A good guideline for how to organize an introduction to your research results.
    \item Michele Tertilt “Writing your paper in LaTex” Lee Crawfurd (2013) “How to make maps” - introduces you to StatPlant, which allows you to quickly make a map showing geographic distribution of summary statistics from an Excel spreadsheet.
\end{itemize}