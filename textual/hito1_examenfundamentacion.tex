
El examen comprensivo de fundamentación busca verificar el nivel de desarrollo de competencia en él área de conocimiento básica de su área de investigación, antes de avanzar a la candidatura doctoral. Dicha competencia se demuestra, en el avance dentro del programa durante los cursos de fundamentación, y se refrenda al aprobar el examen comprensivos en el área básica de conocimiento de su área de interés.

El examen se enmarca en el Marco de Desarrollo de Competencias -MDC- de la Escuela de Economía y Finanzas, contribuyendo a la consolidación de las siguientes áreas de dominio y descriptores:

\begin{figure}[H]
\caption{Articulación del Examen de Fundamentación al MDC \label{hito1_mdc} }
\begin{center}
\includegraphics[width=\textwidth,keepaspectratio]{img/mdc_hito1.png}
\end{center}
\end{figure}

La preparación para este exámenes requiere de un periodo continuo y sostenido de estudio a lo largo de los dos primeros semestres, guiada por cursos secuenciales en microeconomía, macroeconomía, y econometría, los cuales deben ser aprobados con una nota igual o superior a 3.5, de acuerdo al reglamento de posgrado de la Universidad (ver Anexo Institucional 17). Estos cursos constituyen el núcleo de formación en economía y se ofrecen bajo la modalidad magistral. Los estudiantes continúan con su estudio independiente en preparación para los exámenes comprensivos durante el periodo intermedio entre el primer y el segundo semestre.  

Los estudiantes que no aprueben este examen en su primer intento podrán tener una segunda oportunidad. La no aprobación de los dos intentos del examen compresivo de fundamentación conlleva al retiro del estudiante del programa. 

\section{Resultados de aprendizaje}

En el área de dominio Conocimiento y habilidades intelectuales, los resultados de aprendizaje son los siguientes:
\begin{itemize}
\item Planifica, dirige y ejecuta proyectos de investigación innovadores de manera efectiva y sistemática, que contribuyen al avance de la ciencia económica y el mejoramiento de las condiciones de vida de la sociedad.
\item ta competencia se fortalecerá a través de los siguientes descriptores:
\item Construye y ejecuta proyectos de investigación de manera exitosa, garantizando el uso eficiente de los recursos disponibles y la adecuada gestión de los riesgos asociados en su ejecución. ( Nivel de Desarrollo:Introducido-Inicial)
\item Impacta positivamente en el medio a través de actividades de investigación, alta gerencia, consultoría y/o docencia usando una comunicación efectiva hacia la comunidad académica, hacedores de política y líderes empresariales.
\item ta competencia se fortalecerá a través de los siguientes descriptores:
\item Desarrolla estrategias de comunicación y difusión efectivas para los resultados de su área de investigación con el fin de garantizar su impacto en el área de trabajo de investigación. ( Nivel de Desarrollo:Introducido-Inicial)
\end{itemize}

\section{Selección del área de fundamentación}

La selección del área del examen de fundamentación se hará en compañía de la coordinación del doctorado y el asesor(a) en caso de que ya esté definido, los cuales tomarán en cuenta el desempeño de lo estudiante durante los cursos de fundamentación y los temas de interés. 

Las áreas que puede presentarse el examen será en: (i) Microeconomía; (ii) Macroeconomía; y, (iii) Econometría.

\begin{myremark}{Mecanismo de Excepción}
En algunos casos, los estudiantes podrán solicitar la excepción de este examen si demuestran un rendimiento excepcional durante los cursos, en particular: (i) tener un promedio superior a 4.5 en los dos cursos del área seleccionada; (ii) no tener cancelaciones de cursos; y, (iii) no tener homologaciones de universidades externas.
\end{myremark}

\begin{myremark}{Estudiantes de antes de la reforma}
Los estudiantes que están matriculados en el programa antes de la reforma, puede solicitar un cambio de pensum el cual será evaluado por el comité doctoral según el rendimiento del estudiante.
\end{myremark}


\section{Dinámica del Examen}

El examen sigue una metodología del examen basados en problemas donde el estudiante demuestra su conocimiento y comprensión del área del trabajo. Para esto, el examen está compuesto por dos momentos: 

\begin{itemize}
    \item \emph{Examen Escrito} 
    \item \emph{Sustentación Oral}
\end{itemize}

En total tiene una duración de 4 horas. A continuación, se describen los momentos:

\subsection{Momento 1: Examen Escrito (3 Horas)}

El examen escrito es el momento en el cual el estudiante desarrolla con cuidado los procedimientos matemáticos, intuición y desarrollo de las preguntas específicas del examen. 

El examen está compuesto por \textbf{cuatro (4)} preguntas de la siguiente manera:

\begin{itemize}
    \item Dos (2) preguntas del nivel I (e.g. Micro I, Macro I)
    \item Dos (2)  preguntas del nivel II (e.g. Micro II, Macro II)
\end{itemize}

Más que la capacidad de memorizar contenido, el examen busca evaluar la eficacia del estudiante para la preparación y manejo de la información existente. En este sentido, junto al examen el estudiante recibirá un documento de material de estudio preparado previamente por los estudiantes que tomarán el examen comprensivo. 

Con el fin de garantizar que todos los estudiantes tengan la misma información, los estudiantes deberán enviar a la coordinación una copia de los materiales el día previo al examen, los cuales serán distribuidos a todos los estudiantes que presenten el examen. Este material puede incluir: 

\begin{itemize}
    \item Formulas
    \item Mapas conceptuales
    \item Definiciones
    \item Otras herramientas que el estudiante considere necesario.
\end{itemize}

\textbf{Esto es un documento de construcción colectiva. El doctorado no se hace responsable por su contenido}. Por eso, revisa bien y trata de estar seguro(a) que el contenido es correcto.

Durante el examen: 
\begin{itemize}
    \item El estudiante podrá usar calculadora y demás elementos de cálculo.
    \item No se permiten el uso de celulares, computadores, tablets u otra información con acceso a internet.
    \item Después de las 1.5 horas, se hará un descanso de 15 minutos.
    \item Los estudiantes no podrán hablar entre ellos.
\end{itemize}

Al finalizar las 3 horas, el estudiante deberá entregar el documento final.

Como preparación con la Sustentación Oral, el estudiante deberá indicar dos preguntas que defenderá ante el comité de evaluación. La condición es que una pregunta debe corresponder al "Nivel 1" y la otra pregunta debe corresponder al bloque de "Nivel 2". 

\subsection{Momento 2: Sustentación Oral}

Una vez terminada el examen escrito, el estudiante será invitado a una sala donde se encontrarán los profesores que componen el área de conocimiento del examen comprensivo (generalmente los profesores de las materias).

Una vez iniciado el examen, el estudiante recibirá el examen escrito y se iniciará una serie de preguntas relacionadas a los ejercicios seleccionados. Entre el tipo de preguntas que se pueden realizar, se destacan:

\begin{itemize}
 \item Claridad sobre los supuestos
 \item Procedimientos matemáticos
 \item Interpretación de los resultados
\end{itemize}

Se espera que el estudiante demuestre el dominio del tema, de una manera natural y clara. 

\section{Criterio de Evaluación}

El comité de evaluación considerará la parte escrita y oral del estudiante. La evaluación se realizará caso a caso, y se buscará brindar retroalimentación concreta.

El resultado final del examen será: Aprobado o Reprobado.

\section{Necesidades Especiales de Aprendizaje}

Si eres un estudiante con alguna necesidad especial para el aprendizaje (movilidad, espacios, acomodación especial, acompañamiento, entre otros). Por favor, contacta la coordinación del doctorado \modalemail  para hacer los arreglos logísticos y/o adaptación pedagógico necesarios para garantizar el cumplimiento de los objetivos de aprendizaje del examen.

\section{Integridad Académica}

La integridad es el valor supremo de nuestra Universidad, en cuanto le provee horizonte moral a los demás valores institucionales. Este sentido, la integridad constituye un compromiso constante frente a la honestidad, la confianza, la justicia, el respeto, la responsabilidad y el coraje. Incluso en tiempos de adversidad. Se espera que los estudiantes se familiaricen de forma independiente con las políticas de integridad de la Universidad y que reconozcan que su trabajo en el curso debe ser un trabajo propio y original que represente fielmente el tiempo y el esfuerzo aplicados. Las violaciones al Código son muy serias y serán tratadas de una manera que represente plenamente el alcance del Código y que sea acorde con la gravedad de su violación. Si tienes alguna duda,  por favor contactar al centro de integridad de la Universidad EAFIT. 