El periodo de investigación es el más importante del proceso doctoral. Por esta razón, se realiza un seguimientos cercano del avance de la tesis y de la relación estudiante-asesor a través de reportes de avance y autoevaluación, para revisar el cumplimiento de los objetivos de cada curso.

A continuación se incluyen los elementos que deben tener los informes de avance en cada uno de las disertaciones.

\section{Disertación I}

El curso de Disertación I representa el primer momento de la candidatura doctoral, en la cual se espera que el candidato(a) se consolide como un investigador efectivo con altas competencias para generar conocimiento de frontera. En este punto se inicia el proceso de retroalimentación y acompañamiento del parte del asesor el cual acompañará al estudiante en el proceso de realización de su tesis doctoral. En este primer curso, el candidato(a) empezará a perfilar su plan de carrera y se iniciará un proceso basado en la creatividad y el desarrollo de las competencias enmarcadas en el Marco de Desarrollo de Competencias del Doctorado en Economía.

En este sentido, esta asignatura se articula con el el Marco de Desarrollo de Competencias -MDC- de la siguiente manera:

\begin{figure}[H]
\caption{Articulación de Disertación I al MDC \label{hito1_mdc} }
\begin{center}
\includegraphics[width=\textwidth,keepaspectratio]{img/mdc_disertacionI.png}
\end{center}
\end{figure}


Así, el informe de avance debe dar cuenta de los siguientes resultados de aprendizaje:

En el área de dominio Conocimiento y habilidades intelectuales, los resultados de aprendizaje son los siguientes:
\begin{itemize}
\item	Reconoce el valor de los paradigmas de investigación alternativos y es capaz de trabajar y apoyar a otros que trabajan, de una forma interdisciplinaria. (A1.2)
\item	Diseña proyectos que retan el pensamiento tradicional en temas de investigación generales y de progreso. (A2.5)
\item	Toma una actitud creativa e imaginativa respecto a los objetos y metodologías de investigación en economía. (A3.1)
\item	Genera constantemente ideas de investigación novedosas en economía. (A3.3)
\item	Identifica asertivamente la literatura, bases de datos y fuentes de información relevantes en su área de investigación. (A3.4)
\end{itemize}
En el área de dominio Efectividad Personal, los resultados de aprendizaje son los siguientes:
\begin{itemize}
\item	Actúa con iniciativa y motivación, entusiasmo que se transmite a otros miembros de su entorno. (B1.1)
\item	Reconoce sus debilidades, es autocrítico y aprende de sus errores (B1.4)
\item	Persevera y es resiliente frente a obstaculos y retrocesos en su proceso de formación e investigación. (B1.5)
\item	Administra su tiempo de forma eficiente, lo cual le permite cumplir sus compromisos académicos. (B2.1)
\end{itemize}
En el área de dominio Gobernanza de la Investigación, los resultados de aprendizaje son los siguientes:
\begin{itemize}
\item	Establece prioridades para garantizar que su proyecto cumpla con los cronogramas establecidos. (C2.3)
\end{itemize}


\section{Disertación II}

Este curso representa la segunda etapa del proceso de realización de la tesis y la consolidación de los capítulos de la tesis. En este curso, se espera que el candidato(a) desarrolle sus capacidades relacionadas con la gestión del tiempo y la planeación para empiezar a crear un plan de carrera profesional que sea creíble. Al igual que el curso anterior, este proceso es acompañado por el asesor quien le brindará retroalimentación y acompañamiento.

En este sentido, esta asignatura se articula con el el Marco de Desarrollo de Competencias -MDC- de la siguiente manera:

\begin{figure}[H]
\caption{Articulación de Disertación I al MDC \label{hito1_mdc} }
\begin{center}
\includegraphics[width=\textwidth,keepaspectratio]{img/mdc_disertacionII.png}
\end{center}
\end{figure}

Así, el informe de avance debe dar cuenta de los siguientes resultados de aprendizaje:

En el área de dominio Conocimiento y habilidades intelectuales, los resultados de aprendizaje son los siguientes:
\begin{itemize}
\item	Piensa de manera original, independiente y crítica sobre los principales problemas del análisis económico. (A3.2)
\item	Genera constantemente ideas de investigación novedosas en economía. (A3.3)
\end{itemize}
En el área de dominio Efectividad Personal, los resultados de aprendizaje son los siguientes:
\begin{itemize}
\item	Promueve el equilibrio en la vida laboral propia y de los demás, prestando especial atención a las señales de exceso de presión y estrés. (B2.2)
\item	Formula un plan creíble de carrera que gestiona a través del buen desempeño profesional y las redes (B3.3)
\end{itemize}
En el área de dominio Compromiso e Influencia en el Medio, los resultados de aprendizaje son los siguientes:
\begin{itemize}
\item	Reconoce su participación real en el logro de resultados grupales. (D1.4)
\end{itemize}

\section{Disertación III}

Este curso representa la tercera etapa del proceso de realización de la tesis. Se espera que el candidato(a) ya tenga un avance significativo de su tesis con un primer artículo de segundo año maduro, un segundo artículo en un estado avanzado y el desarrollo del último capítulo de la tesis. En este curso, se espera que el estudiante desarrolle sus capacidades relacionadas con la gestión del tiempo, la planeación y empiece a crear un plan de carrera profesional que sea creíble. Al igual que el curso anterior, este proceso es acompañado por el asesor quien le brindará retroalimentación y acompañamiento.

En este sentido, esta asignatura se articula con el el Marco de Desarrollo de Competencias -MDC- de la siguiente manera:

\begin{figure}[H]
\caption{Articulación de Disertación I al MDC \label{hito1_mdc} }
\begin{center}
\includegraphics[width=\textwidth,keepaspectratio]{img/mdc_disertacionIII.png}
\end{center}
\end{figure}

Así, el informe de avance debe dar cuenta de los siguientes resultados de aprendizaje:

En el área de dominio Conocimiento y habilidades intelectuales, los resultados de aprendizaje son los siguientes:
\begin{itemize}
\item	Es autocrítico, busca nuevas maneras de mejorar su desempeño y se esfuerza por realizar investigación de excelencia en economía. (A2.2)
\end{itemize}
En el área de dominio Efectividad Personal, los resultados de aprendizaje son los siguientes:
\begin{itemize}
\item	Reconoce sus debilidades, es autocrítico y aprende de sus errores (B1.4)
\end{itemize}
En el área de dominio Gobernanza de la Investigación, los resultados de aprendizaje son los siguientes:
\begin{itemize}
\item	Tiene conocimientos sobre administración de proyectos de investigación (C2.2)
\item	Gestiona de manera adecuada y eficiente los diferentes recursos disponibles en los proyectos de investigación que participa. (C3.2)
\end{itemize}
En el área de dominio Compromiso e Influencia en el Medio, los resultados de aprendizaje son los siguientes:
\begin{itemize}
\item	Expresa alta motivación y compromiso para colaborar en trabajos con el supervisor y otros compañeros del doctorado (D1.2)
\item	Reconoce la importancia de responder por sus acciones respecto a los impactos económicos y sociales de su investigación. (D3.2)
\end{itemize}

\section{Disertación IV}

Este curso representa el último momento del proceso de realización de la tesis. En este punto, se espera que el candidato(a) ya tenga su tesis completa y que se dedique a su finalización, corrección y envío de al menos un artículo a una revista científica siguiendo el reglamento del Doctorado. Se espera que el estudiante entienda las implicaciones éticas y de posible impacto de su trabajo. En este punto, el candidato (a) escriba la introducción y conclusiones del trabajo y adapte el trabajo para su presentación. Al igual que el curso anterior, este proceso es acompañado por el asesor quien le brindará retroalimentación y acompañamiento.

En este sentido, esta asignatura se articula con el el Marco de Desarrollo de Competencias -MDC- de la siguiente manera:

\begin{figure}[H]
\caption{Articulación de Disertación I al MDC \label{hito1_mdc} }
\begin{center}
\includegraphics[width=\textwidth,keepaspectratio]{img/mdc_disertacionIV.png}
\end{center}
\end{figure}

Así, el informe de avance debe dar cuenta de los siguientes resultados de aprendizaje:

En el área de dominio Conocimiento y habilidades intelectuales, los resultados de aprendizaje son los siguientes:
\begin{itemize}
\item	Diseña proyectos que retan el pensamiento tradicional en temas de investigación generales y de progreso. (A2.5)
\item	Toma una actitud creativa e imaginativa respecto a los objetos y metodologías de investigación en economía. (A3.1)
\end{itemize}
En el área de dominio Efectividad Personal, los resultados de aprendizaje son los siguientes:
\begin{itemize}
\item	Actúa con iniciativa y motivación, entusiasmo que se transmite a otros miembros de su entorno. (B1.1)
\item	Actúa con integridad profesional y honestidad en las diferentes actividades académicas, de investigación y otras relacionadas a su actividad profesional. (B1.2)
\item	Persevera y es resiliente frente a obstaculos y retrocesos en su proceso de formación e investigación. (B1.5)
\item	Administra su tiempo de forma eficiente, lo cual le permite cumplir sus compromisos académicos. (B2.1)
\item	Promueve el equilibrio en la vida laboral propia y de los demás, prestando especial atención a las señales de exceso de presión y estrés. (B2.2)
\item	Formula un plan creíble de carrera que gestiona a través del buen desempeño profesional y las redes (B3.3)
\end{itemize}
En el área de dominio Gobernanza de la Investigación, los resultados de aprendizaje son los siguientes:
\begin{itemize}
\item	Establece prioridades para garantizar que su proyecto cumpla con los cronogramas establecidos. (C2.3)
\end{itemize}
En el área de dominio Compromiso e Influencia en el Medio, los resultados de aprendizaje son los siguientes:
\begin{itemize}
\item	Reconoce la importancia de la retroalimentación, tanto de su supervisor como de otros profesores, y tiene la capacidad de dar retroalimentación a colegas. (D1.1)
\item	Le apunta a la más prestigiosa publicación en salidas académicas y no académicas. (D2.3)
\item	identifica el impacto económico, legal y social que implica el uso de la información y la maneja de manera ética y responsable (D3.1)
\end{itemize}