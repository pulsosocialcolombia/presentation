La defensa de tesis constituye el hito más importante del Doctorado, porque es en este momento donde se materiliza el desarrollo de todas las competencias del Marco de Desarrollo de Competencias del Doctorado en Economía. En este sentido, se espera que los estudiantes logren demostrar que han desarrollado todas las competencias del perfil de egreso del programa.

El curso se enmarca en el Marco de Desarrollo de Competencias -MDC- de la Escuela de Economía y Finanzas, contribuyendo a la consolidación de las siguientes áreas de dominio y descriptores:

\begin{figure}[H]
\caption{Articulación de la pasantía internacional al MDC\label{hito1_mdc} }
\begin{center}
\includegraphics[width=\textwidth,keepaspectratio]{img/mdc_hito6.png}
\end{center}
\end{figure}

De esta manera, al final el curso lo estudiantes habrán fortalecido las siguientes competencias:

\section{Resultados de Aprendizaje}


En el área de dominio Conocimiento y habilidades intelectuales, los resultados de aprendizaje son los siguientes:
\begin{itemize}
\item 	Utiliza las principales metodologías de investigación adecuadas conforme a sus preguntas de investigación. (A1.1)
\item 	Reconoce el valor de los paradigmas de investigación alternativos y es capaz de trabajar y apoyar a otros que trabajan, de una forma interdisciplinaria. (A1.2)
\item 	Construye argumentos sólidos, con premisas explícitas y conclusiones que se desprenden lógicamente de las premisas (A2.3)
\item 	Piensa de manera original, independiente y crítica sobre los principales problemas del análisis económico. (A3.2)
\end{itemize}
En el área de dominio Efectividad Personal, los resultados de aprendizaje son los siguientes:
\begin{itemize}
\item 	Confía en sus propias habilidades, lo cual le permite defender exitosamente sus ideas, incursionar en nuevos desafíos en su área de investigación (B1.3)
\item 	Lidera y participa en redes, dentro y fuera de la academia, para aumentar la visibilidad y el impacto de su investigación. (B3.3)
\end{itemize}
En el área de dominio Gobernanza de la Investigación, los resultados de aprendizaje son los siguientes:
\begin{itemize}
\item 	Hace uso ético y responsable de las tecnologías de la información y la comunicación (C1.3)
\item 	Almacena, administra y maneja de forma efectiva información, datos y referencias bibliográficas usando metodologías y herramientas apropiadas. (C2.1)
\item 	Demuestra en sus investigaciones un dominio de los estándares de calidad disciplinar propios de revistas científicas pertenecientes a los cuartiles Q1 y Q2 de los índices bibliográficos a nivel internacional (C2.4)
\end{itemize}
En el área de dominio Compromiso e Influencia en el Medio, los resultados de aprendizaje son los siguientes:
\begin{itemize}
\item 	Hace un uso positivo de la diversidad, la retroalimentación y la diferencia para enriquecer proyectos y resultados de investigación. (D1.3)
\item 	Presenta su trabajo con cuidado y rigor científico usando un lenguaje adecuadad y un buen manejo del tiempo en los diferentes escenarios académicos (D2.1)
\item 	Le apunta a la más prestigiosa publicación en salidas académicas y no académicas. (D2.3)
\item 	Es capaz de comunicar efectivamente la investigación a una audiencia diversa y elabora material escrito de diversos tipos (informe, ensayo, acta) con coherencia, claridad y precisión, reconociendo la intención comunicativa y el público al que va dirigido (D2.4)
\item 	Reconoce la importancia de responder por sus acciones respecto a los impactos económicos y sociales de su investigación. (D3.2)
\item 	Su actitud es la de un ciudadano global inmerso en la economía internacional y responsable por realizar contribuciones que expandan la frontera del conocimiento en economía. (D3.4)
\item 	Tiene la habilidad de llevar el conocimiento de la investigación hacia el proceso de formulación de política a través de una variedad de mecanismos. (D3.5)
\end{itemize}




\section{Dinámica de la Defensa}

El proceso de constitución de tribunal está regido por el Artículo 32 del Reglamento de Doctorado en Economía. Dicho proceso está compuesto por cuatro momentos. 

En total, el proceso puede tardarse hasta 5.5 meses, desde la solicitud de defensa por parte del estudiante. 

Nota: Desde la coordinación se harán todos los procedimientos para reducir este tiempo en la medida de lo posible.

Los cuatro momentos son:

\includegraphics[scale=0.5]{img/timeline_defensa.png}

\section{Paso 1: Solicitud de Defensa}

El proceso de inicia con la solicitud de parte del estudiante y del asesor. Esta petición se formalizará a través del envío de los siguientes documentos:

\begin{itemize}
    \item \emph{Versión completa de la tesis}
        \begin{itemize}
            \item Seguir los parámetros de formato mínimos requeridos (para Ms Word se cuenta con una plantilla TesisDoctoralEAFIT\_Plantilla)
            \item Se requiere que la tesis contenga, como mínimo, las siguientes secciones:
            \begin{itemize}
                \item Introducción 
                \item Capitulo 1: Articulo 1
                \item Capitulo 2: Articulo 2
                \item Capitulo 3: Articulo 3
                \item Conclusiones/Discusión
                \item Referencias
            \end{itemize}
            \item Se deber recordar que por lo menos uno de los artículos debe estar escrito en inglés (Artículo 12 del Reglamento del Doctorado).
        \end{itemize}   
        
    \item \emph{Informe de actividades del estudiante aprobado por el asesor} En este informe el estudiante debe describir de manera detallada, y con evidencias, las diferentes actividades que realizó durante su proceso doctoral.  En particular: 
     \begin{itemize}
            \item Publicaciones.
            \item Participación en eventos académicos.
            \item Pasantía de investigación o la movilidad académica (aquí puede usar el informe entregado para avalar la pasantía).
            \item Resultados de Exámenes compresivos.
            \item Aprobación de las materias del pensum.
            \item Tener aprobadas y confirmadas el 100\% de las notas de todas las materias del pensum que le corresponde, puede consultar su plan académico en Ulises.
            \item Acreditar suficiencia en una segunda lengua en el idioma inglés con un nivel (B2) según los estándares establecidos por el Marco Común Europeo (MCE), avalado por un examen de suficiencia reconocido internacionalmente de acuerdo a las políticas establecidas por la Dirección de Idiomas de la Universidad EAFIT.
            \item Cambios importantes con respecto a la propuesta de tesis aprobada.
            \item Otros aspectos importantes para demostrar el cumplimiento de los requisitos doctorales.
    \end{itemize}
    
    \item \emph{Propuesta de Tribunal}
    \begin{itemize}
        \item El director de tesis del estudiante recomendará por lo menos seis (6) jurados, de los cuales por lo menos dos (2) deberán ser internacionales, para realizar la evaluación de la tesis.
        \item El director deberá mencionar los tres (3) jurados que consideraría los titulares (ideales), de los cuales por lo menos un (1) debe ser internacional.  Los otros 3, serán usados en caso de que no ser posible contactar los iniciales. 
        \item Todos los miembros del tribunal deberán tener título doctorado y ser investigadores reconocidos en el área de estudio.
    \end{itemize}
    
\end{itemize}

Los documentos serán revisados por la Coordinación del Doctorado en Economía y, en caso de tener alguna inconsistencia, serán devueltos al estudiante hasta que se cumpla con los todos los requisitos y proceder al siguiente paso.

\section{Paso 2: Constitución del Tribunal}

La solicitud de defensa de tesis será presentada por la Coordinación al Comité Doctoral para su consideración y aprobación. En este proceso, se revisará el cumplimiento de todos los requerimientos para la defensa de tesis usando las evidencias y actividades del informe de actividades.

Una vez aprobado los requisitos, el Comité de Doctorado en Economía, procederá a designar tres (3) jurados, por lo menos uno (1) será internacional.

\section{Paso 3: Evaluación Tribunal}

Una vez se hayan designado los miembros del tribunal doctoral, la Coordinación iniciará el proceso de contacto con el siguiente protocolo:
\begin{itemize}
    \item Envío de e-mail de invitación.  
    \item En caso de que no se tenga respuesta, se enviarán dos recordatorios cada semana.
    \item Si no se recibe ninguna confirmación por parte del jurado invitado, se procederá con el siguiente en la lista. 
\end{itemize}

El proceso de contacto y confirmación puede tardar hasta 1.5 meses.

Una vez se tenga la confirmación, cada miembro del tribunal recibirá la tesis completa. A partir de este momento, los jurados dispondrán de un plazo máximo de dos (2) meses para su lectura y realización de informe de evaluación.

Los jurados evaluarán el documento y emitirán un concepto dirigido al Comité de Doctorado en Economía, haciendo las precisiones, críticas y reflexiones suscitadas por el texto. 

En caso de que se requiera, el estudiante deberá realizar las modificaciones pertinentes en su tesis de acuerdo con las recomendaciones emitidas por los jurados.

El estudiante tendrá hasta un 1 mes para responder los puntos cruciales de la evaluación.

\section{Paso 4: Defensa Pública}

Con la aprobación final del tribunal doctoral, el Comité de Doctorado en Economía fijará la fecha para que el estudiante realice la defensa de la tesis.

La defensa pública tendrá una duración entre dos (2) a cuatro (4) horas y será un acto público de invitación abierta.

Para la defensa se tendrá la siguiente estructura:

\begin{itemize}
    \item 	Constitución del tribunal
    \begin{itemize}
        \item La Coordinación del Doctorado verificará la presencia del tribunal y certificará su constitución.
        \item Previamente, la Coordinación designará al jurado tres roles:
        \begin{itemize}
            \item Presidente del tribunal
            \item Secretario del tribunal
            \item Vocal del tribunal
        \end{itemize}
        \item El presidente del tribunal dará paso a la presentación del estudiante. La presentación tendrá una duración de 40 minutos máximo.
        \item El presidente dará la palabra a los miembros del tribunal quiénes preguntarán al estudiante sobre toda su tesis.
        \item Una vez terminada la intervención del tribunal, se abrirá la oportunidad para que algún miembro del público, con doctorado, pueda hacer preguntas.
        \item Posteriormente, el tribunal se retirará de la sala para deliberar.
        \item Después de deliberar, retornan a la sala y darán su concepto: 
        \item La tesis doctoral tendrá una calificación única por parte del jurado, de carácter cualitativo, de \emph{Aprobada} o \emph{Reprobada}.
    \end{itemize}
        
\end{itemize}

Una vez aprobado, se realiza la proclamación del nuevo o nueva doctor o doctora en Economía.

Adicionalmente, el jurado de tesis podrá recomendar al Comité de Doctorado en Economía, mediante sustentación escrita, el otorgamiento de Mención de honor (Summa Cum Laude). El otorgamiento de la Mención de Honor se hará por iniciativa de algunos de los integrantes del jurado de tesis, bajo una exposición de motivos escrita, basada en criterios de alta calidad y contribución significativa, y \emph{aprobada por unanimidad del jurado de tesis}.

En caso de reprobación de la tesis finalizada, el estudiante podrá solicitar una segunda oportunidad para defenderla en un período no inferior a seis (6) meses ni superior a un (1) año, contado a partir de la fecha en que recibió la reprobación. En caso de no aprobar la tesis doctoral en la segunda oportunidad, el Comité de Doctorado en Economía, en conjunto con el director de tesis, decidirá sobre la permanencia del estudiante en el Programa de Doctorado en Economía.

\section{Paso 5: Graduación Administrativa}

Una vez terminada el proceso académico de graduación, el estudiante iniciará el proceso administrativo para participar en las ceremonias institucionales de graduación.  

Para optar al título de Doctor en Economía debe cumplir con los paz y salvos en las fechas indicadas por Admisiones y Registro para cada ceremonia de grado. Cada una de las dependencias delegadas se encargarán de verificar y confirmar su cumplimiento.

\begin{itemize}
    \item \textbf{Dependencia:} Admisiones y Registro
        \begin{itemize}
            \item Paz y Salvo Académico: Tener aprobadas y confirmadas el 100\% de las notas de todas las materias del pensum que le corresponde, puede consultar su plan académico en Ulises
            \item Paz y Salvo Documentación: o	Si se encuentra con documentos pendientes se le informará por parte de Admisiones y Registro al correo institucional, si todo está en regla no se le notificará.
        \end{itemize}
        
    \item \textbf{Dependencia:} Tesorería y cartera
            \begin{itemize}
            \item Paz y Salvo Financiero : No tener deudas pendientes con la Universidad.
        \end{itemize}
        
    \item \textbf{Dependencia:} Biblioteca 
            \begin{itemize}
            \item Paz y Salvo Financiero
             \begin{itemize}
                 \item Debe haber entregado en la Biblioteca de la Universidad EAFIT Medellín su Trabajo de Grado, Monografía o Tesis, si aplica para su programa. 
                 \item No debe tener sanciones pendientes
                 \item No debe tener material prestado.
             \end{itemize}
        \end{itemize}
        
\end{itemize}

La entrega de la Tesis de Grado lo realiza cada estudiante a través del módulo de Autoarchivo habilitado desde la Biblioteca, para lo cual se relacionan los accesos requeridos:

\begin{itemize}
    \item \href{http://www.eafit.edu.co/biblioteca/servicios/Documents/creacion_archivos_pdf.pdf}{Guía para la creación de archivos PDF} : explica cómo poner metadatos al documento de Word y como convertir el archivo en formato PDF/A.
    \item \href{http://www.eafit.edu.co/biblioteca/busqueda-servicios/Documents/carta_aprobacion_trabajo_grado_eafit.docx}{Carta de aprobación de tesis de grado}: este es un documento legal, por tanto, no debe tener enmendaduras y debe estar firmado por el asesor.
    \item \href{http://www.eafit.edu.co/biblioteca/busqueda-servicios/Documents/formulario_autorizacion_publicacion_obras.pdf}{ Formulario de autorización de publicación de obras}: el formulario debe estar firmado por el(los) autor(es) de la tesis de grado y no debe tener enmendaduras. Es requisito indispensable que todos los autores firmen este documento. Abrir con Acrobat Reader para diligenciar.
    \item \href{http://hdl.handle.net/10784/12512}{ Instructivo autoarchivo}: Repositorio Institucional

\end{itemize}   