El plan general de estudios del Doctorado en Economía se estructura alrededor de la investigación en sentido estricto cuyo objetivo es la creación de nuevo conocimiento en economía y la búsqueda de solución a problemas económicos y sociales de nuestro país y del mundo. La Universidad EAFIT considera la investigación como la mejor fuente de mejoramiento de sus programas académicos, tanto de pregrado como de posgrado, y de su personal docente y discente, y la realiza como una actividad intencionalmente planeada y articulada en el marco del cumplimiento de su misión y del logro de su visión de consolidarse como una universidad de docencia con investigación \footnote{En el anexo \ref{anex:anexo1} se encuentra una lista de referencias bibliográficas que pueden ser útiles para la preparación del primer semestre del doctorado.}. 

En los dos primeros años del Doctorado, los estudiantes desarrollan competencias fundamentales en teoría económica y sus aplicaciones. A partir del tercer semestre el Doctorado cambia su foco hacia los cursos electivos de énfasis, la escritura y el desarrollo de la agenda de investigación del estudiante.  El plan de estudios se articula alrededor de tres ciclos principales: fundamentación, profundización e investigación. En el primero, se centra en el desarrollo de competencias relacionadas a la primera dimensión del modelo de competencias, conocimiento y habilidades intelectuales, concentrándose en el estudio del núcleo moderno de la economía en sus respectivas áreas: microeconomía y macroeconomía; así como en las competencias cuantitativas fundamentales para el análisis económico moderno y el diseño e implementación de estudios empíricos rigurosos. Posteriormente, en el ciclo de profundización, el estudiante podrá concentrarse en su área de interés, tomando una serie de electivas enmarcadas en su interés de investigación; en este punto, el estudiante empieza a desarrollar de manera más integral las cuatro dimensiones del marco de competencias del doctorado. Finalmente, en el ciclo de investigación, el estudiante debe concentrarse en el diseño y ejecución de su tesis doctoral, la cual representa la materialización de todas las competencias relacionadas con el doctorado.

Con el fin de asegurarse sobre el desarrollo de los resultados de aprendizaje esperados en cada uno de los ciclos, cada uno tiene un hito o actividad clave. Estas actividades permiten reforzar y demostrar, las habilidades obtenidas una vez se superan los requisitos académicos de cada uno de los ciclos. Cada uno de ellos se discutirá en detalle en la próxima sección. La tabla 3.3 presenta el esquema secuencial del plan de estudios del Doctorado en Economía, donde los colores significan los diferentes ciclos.


\begin{figure}[H]
\caption{Esquema General Plan de Estudios del Doctorado en Economía \label{Estructura} }
\begin{center}
\includegraphics[width=\textwidth,keepaspectratio]{img/Estructura_Curricular.png}
\end{center}
\end{figure}

De esta manera, se espera que el estudiante doctoral pueda desarrollar competencias específicas propias de cada ciclo y, de manera paralela, pueda ser expuesto a otras competencias transversales. Es decir, los contenidos curriculares y las actividades evaluativas en cada actividad permiten evaluar una competencia específica, pero al mismo tiempo expone a otras de manera indirecta.  Con el fin de evaluar el nivel de desarrollo de los resultados de aprendizaje, la Universidad EAFIT ha adoptado un sistema de clasificación que permite mostrar el nivel de aprendizaje que se espera que el alumno alcance en una escala de dominio o competencia al final del curso. Esto combina tanto el modelo de Drefus y la taxionomía de Bloom, discutidas anteriormente. A saber, los tres niveles son los siguientes:

\begin{itemize}
    \item \emph{Introducido-Inicial}: Se espera que el estudiante tenga un acercamiento introductorio al resultado de aprendizaje, ya sea de manera directa o indirecta. En particular, se espera el desarrollo a nivel de principiante entre novato-principiante avanzado (modelo de Drefus), donde el estudiante conozca y entienda los principios del resultado de aprendizaje (modelo de Bloom). Instala la competencia (modelo ACSB).
    \item \emph{Expuesto-Medio}: En un nivel medio, se asume que el estudiante conoce y entiende el resultado de aprendizaje. De este modo, se espera que al finalizar el curso el estudiante sea competente (modelo de Drefus) para aplicar de manera acertada el resultado de aprendizaje en un problema específico y un contexto determinado. Refuerza la competencia (modelo ACSB).
    \item \emph{Evaluado-Avanzado}: El estudiante es expuesto y evaluado de manera rigurosa al resultado de aprendizaje, con el objetivo que sea capaz y esté cualificado para analizar críticamente dicho resultado. Esto le permitirá evaluar y crear nuevas ideas y conjeturas a partir de la comprensión profunda del resultado de aprendizaje. Se evidencia la competencia (modela ACSB).

\end{itemize}


En términos del modelo de tunning, se espera que las competencias transversales se refieren a los dos primeros niveles (introducido-inicial y expuesto-medio) y las especificas al último (evaluado-avanzado). Es decir, las actividades evaluativas darán evidencias pedagógicas de aquellos resultados de aprendizaje que tratados de manera directa, mientras otras competencias transversales son tratadas de manera indirecta y no se recogen evidencias evaluativas. El siguiente gráfico resume el nivel de desarrollo de las competencias según el MDC. 


\begin{figure}[H]
\caption{Nivel de desarrollo de las competencias según el ciclo de formación del Doctorado en Economía de la Universidad EAFIT \label{Estructura} }
\begin{center}
\includegraphics[width=\textwidth,keepaspectratio]{img/Nivel_Desarrollo.png}
\end{center}
\end{figure}

A continuación, se hace una descripción general del cada uno de los ciclos, la organización de las actividades académicas, lo hitos y el nivel de desarrollo esperando de los resultados de aprendizaje. 

\section{Ciclo de Fundamentación}

Los cursos obligatorios se diseñaron de tal forma que sus contenidos reflejen el núcleo moderno de la economía en las dos principales áreas de conocimiento: microeconomía y macroeconomía. A través de dos cursos avanzados, se espera que los estudiantes puedan construir un entendimiento profundo de los pilares fundamentales de la microeconomía y la macroeconomía moderna y comprender las conexiones entre los mismos. Adicionalmente, dada la importancia de la modelación en la economía moderna, existen tres cursos obligatorios en métodos cuantitativos: matemáticas avanzadas para economía y dos cursos de econometría. El curso de matemáticas avanzadas se diseñó para asegurar que los estudiantes adquieran las competencias y las herramientas fundamentales para desempeñarse exitosamente en los cursos de microeconomía, macroeconomía y econometría. Se entiende que para algunos estudiantes el curso tendrá como principal propósito servir de repaso, pero para la mayoría dicho curso será fundamental para empezar a construir su portafolio de herramientas requeridas para la investigación. Los cursos de econometría buscan dotar a los estudiantes de los elementos teóricos más relevantes y las herramientas necesarias para llevar a cabo estudios empíricos con rigor, de tal forma que sus conclusiones puedan ser usadas en el diseño y evaluación de políticas económicas. 

En términos de resultados del desarrollo de aprendizaje, se espera que durante el ciclo de profundización se aborden 23 resultados de aprendizaje del área de dominio de conocimientos y habilidades intelectuales. Asimismo, se espera que se aborden de manera indirecta otras competencias que son desarrolladas a nivel inicial o intermedio. El siguiente gráfico representa los resultados. 

\begin{figure}[H]
\caption{Nivel de desarrollo de las competencias según el ciclo de formación del Doctorado en Economía de la Universidad EAFIT \label{Estructura} }
\begin{center}
\begin{tabular}{p{0.5\textwidth}|p{0.5\textwidth}}
   \textbf{Competencias Específicas}
Resultados de aprendizaje evaluados
  & \textbf{Competencias Transversales} (expuestos de manera indirecta)  \\
  \includegraphics[width=0.5\textwidth,keepaspectratio]{img/funda_1.png}   &  \includegraphics[width=0.5\textwidth,keepaspectratio]{img/funda_2.png}
\end{tabular}

\end{center}
\end{figure}

El ciclo de fundamentación se cierra con el primer hito del doctorado: el exámenes compresivos de fundamentación. 

\section{Ciclo de Profundización}

La asignación de trabajo se intensifica en el tercer semestre del Doctorado en el cual se deben tomar cursos de énfasis en las áreas de microeconomía, macroeconomía, o finanzas de acuerdo a lo intereses investigativos del estudiante. Estas líneas están apoyadas por las líneas de investigación de la Escuela de Economía y Finanzas . Los cursos electivos constituyen el primer componente de flexibilización del programa. Los mismos se podrán configurar de acuerdo a los intereses particulares de los estudiantes y a la oferta académica de la Universidad. Dichos cursos ofrecerán a los estudiantes la posibilidad de profundizar sus conocimientos en una o dos áreas de su interés entre microeconomía, macroeconomía, finanzas, o econometría. Los cursos se ofrecerán teniendo en cuenta la capacidad institucional de apoyar investigaciones de nivel doctoral en campos específicos dentro de dichas áreas. Los cursos electivos por cada línea de profundización son los siguientes: 

\begin{itemize}
\item 	\emph{Línea de Profundización en Finanzas}: Teoría financiera I y II, Valoración de Activos, Cálculo Estocástico para Finanzas, Riesgo Financiero, y Finanzas Empíricas (Econometría Financiera Aplicada).
\item 	\emph{Línea de Profundización en Macroeconomía}: Crecimiento y Desarrollo Económico, Modelos de Equilibrio General Estocástico, Inequidad y Desarrollo Económico, Economía Internacional, Economía Espacial, Macro-econometría.
\item 	\emph{Línea de Profundización en Microeconomía}: Economía Laboral, Organización Industrial, Economía del Sector Publico, Evaluación de Impacto, Economía Espacial, Microeconometría.
\end{itemize}

Las lista no se limita a estas opciones. Dado las especificidad de algunas áreas de estudios, existe la posibilidad de que se creen cursos específicos bajo el nombre de tópicos, el cual permite una alta flexibilidad para la profundización de parte de los estudiantes. Adicionalmente, las asignaturas electivas podrán cursarse tanto en la Universidad EAFIT, como en una Universidad nacional o extranjera, previa aprobación del Comité Doctoral. La Universidad con los convenios de movilidad con estos fines con la Universidad Católica de Lovaina, American University, Brandeis University, y Carlos III de Madrid. En los próximos capítulos se describen dichos acuerdos en detalle. Igualmente, siguiendo la tradición de las maestrías en economía y en finanzas, se ofrecerán otros cursos electivos que serán dictados por profesores extranjeros en las escuela de verano.

Cada estudiante deberá elegir un campo de especialización. Los campos de microeconomía, macroeconomía y finanzas comprenden tres cursos electivos más un curso electivo complementario que podrá ser de otro campo a elección del estudiante y de acuerdo a la oferta disponible por parte de la Universidad.  Los cursos electivos se ofrecen bajo la modalidad magistral combinada con la modalidad seminarios o bajo la modalidad de seminarios exclusivamente dependiendo de la familiaridad de los estudiantes con el campo de estudio. Algunos cursos se ofrecen bajo la modalidad de trabajo dirigido en función del número de estudiantes. Los cursos tienen como objetivo explorar los últimos desarrollos en el campo de conocimiento correspondiente de tal forma que los estudiantes identifiquen el estado del conocimiento en dicha área y puedan anticipar qué tipo de contribuciones podrían ser de interés para la economía en general y para el campo de estudio en particular. Los cursos se ofrecen de acuerdo a los intereses y las agendas de investigación de los estudiantes de tal forma que se garantice una adecuada formación en investigación y se mantenga un adecuado balance económico. 

Desde el punto del desarrollo de las competencias a partir de los resultados de aprendizaje, se observa que aún prevalece la dimensión de conocimiento y habilidades intelectuales, con siete resultados de aprendizaje, lo que implica que se espera que el estudiante desarrolle la fundamentación disciplinar, habilidades cognitivas y creatividad propia de su área de investigación. No obstante, a diferencia del primer ciclo, aquí también se recogen evidencias de otros resultados de aprendizaje relacionados con el compromiso con el medio y la efectividad personal. Por último, en cuanto a las competencias transversales, se prevalece se inicia a tener una mayor exposición en los demás áreas de dominio (ver el siguiente gráfico).

\begin{figure}[H]
\caption{Estructura de competencias del ciclo de profundización según el modelo MDC \label{Estructura} }
\begin{center}
\begin{tabular}{p{0.5\textwidth}|p{0.5\textwidth}}
   \textbf{Competencias Específicas}
Resultados de aprendizaje evaluados
  & \textbf{Competencias Transversales} (expuestos de manera indirecta)  \\
  \includegraphics[width=0.5\textwidth,keepaspectratio]{img/profu_1.png}   &  \includegraphics[width=0.5\textwidth,keepaspectratio]{img/profu_2.png}
\end{tabular}

\end{center}
\end{figure}

El ciclo de profundazación se cierra con el segundo hito del doctorado: Examen de profundización en el campo de especialización. 

\section{Ciclo Investigativo}

Una vez terminado los ciclos de fundamentación y profundización, el estudiante inicia su proceso de formación de investigación de la fase doctoral con la realización del trabajo doctoral. La disertación doctoral es escrita bajo la dirección, supervisión, y atención cercana de un tutor, el cual es identificado por parte de estudiante y la coordinación del doctorado y aceptado por el Comité Doctoral.  Dicho tutor podrá ser cambiado a petición del estudiante con el fin de lograr una mejor correspondencia entre los intereses del estudiante y el director final de su disertación doctoral. Cualquier cambio deberá ser solicitado por escrito al Comité del Doctorado y, en todo caso, deberá realizarse antes del inicio del cuarto semestre dentro del programa. El tutor y el estudiante, desde el primer semestre del estudiante en el programa deberán realizar un plan de progreso a través del programa para el estudiante para asegurar que todos los requisitos se cumplan con la oportunidad necesaria y que la agenda de investigación del estudiante progrese adecuadamente hacia la fase de disertación.

Durante el cuarto semestre se inicia formalmente el ciclo de investigación con los cursos de fundamentos de investigación y la escritura del artículo de segundo año. Para cumplir dicho requisito, los estudiantes deberán preparar y presentar un artículo de investigación, el cual constituye el insumo principal para la propuesta de tesis y su postulación a la candidatura doctoral (hito 3) que se discute a continuación. Una vez se obtiene la candidatura, el candidato(a) inicia el proceso de realización de sus tesis a través de las materias de disertación y de consolidación de las competencias investigativas con los cursos de seminario.

Desde el punto de vista de los resultados de aprendizaje evaluados, se encuentra un incremento importante de resultados de aprendizaje evaluados. En total, se recogen evidencias sobre 47 resultados de aprendizaje durante el proceso en todas las áreas de dominio. Durante este proceso la formación se centra en la consolidación de aspectos relacionados con el pensamiento crítico, la capacidad de gestionar y liderar proyectos bajo la dimensión de gobernanza de la investigación. Además, se espera que el director de la tesis acompañe al estudiante en la formación de su carrera profesional e identificación de la formación complementaria para cumplir con el marco de competencias propuesto para la formación doctoral. El siguiente gráfico resume la estructura.

\begin{figure}[H]
\caption{Estructura de competencias del ciclo de investigación según el modelo MDC \label{Estructura} }
\begin{center}
\begin{tabular}{p{0.5\textwidth}|p{0.5\textwidth}}
   \textbf{Competencias Específicas}
Resultados de aprendizaje evaluados
  & \textbf{Competencias Transversales} (expuestos de manera indirecta)  \\
  \includegraphics[width=0.5\textwidth,keepaspectratio]{img/inv_1.png}   &  \includegraphics[width=0.5\textwidth,keepaspectratio]{img/inv_2.png}
\end{tabular}

\end{center}
\end{figure}

Durante este período existen diferentes hitos que constituyen la esencia misma de la formación doctoral. El primero, hace referencia a la propuesta de tesis, la cual permite que estudiante presente formalmente su aspiración a la candidatura doctoral, a continuación se resume el proceso.

Cada semestre a partir de su admisión a la candidatura doctoral, deberán presentar informes escritos sobre sus avances. Cada informe, deberá ser aprobado por el Director del Comité de Disertación del estudiante y el Director del Doctorado. El informe tendrá una calificación de aprobado o reprobado. En caso de reprobar alguno de dichos informes, la razones para no ser aprobado deberán ser subsanadas antes de la presentación del informe subsiguiente.

Como requisito para el progreso en la escritura de la disertación doctoral, los estudiantes deberán matricular y asistir los seminarios de investigación I, II y III. Estos cursos tienen tres objetivos dentro de la formación: (1) consolidar las competencias investigativas a través de cursos virtuales en temas relacionados con la ética en la investigación, técnicas de presentación y gestión de la información; (2) asistir a todos los eventos académicos del doctorado con el fin de mejorar sus habilidades investigativas y ampliar sus redes académicas; (3) avances de su disertación en el Seminario doctoral de la Escuela de Economía y Finanzas. En particular, los tres seminarios tienen la siguiente estructura:

\begin{itemize}
\item 	\emph{Seminario I}: El Seminario doctoral I se enmarca en el modelo de competencias de formación para la investigación de la Universidad EAFIT, el cual busca consolidar las competencias transversales en investigación de los estudiantes de posgrados de investigación. En este primer seminario, la formación se enfoca en la consolidación de las competencias informacionales y digitales, a través de la combinación de un curso aplicado en escritura académica y la realización del "Curso virtual en competencias informacionales y digitales" liderado por el Centro Cultural Biblioteca Luis Echavarría Villegas. Al final de la asignatura se espera que los estudiantes cuenten con las herramientas y las capacidades para discernir sobre el uso adecuado de las diferentes fuentes de información.
\item 	\emph{Seminario II}: El Seminario doctoral II se enmarca en el modelo de competencias de formación para la investigación de la Universidad EAFIT, el cual busca consolidar las competencias transversales en investigación de los estudiantes de posgrados de investigación. En este segundo seminario, la formación se enfoca en la consolidación de las competencias relacionadas con la comunicación de la ciencia, las cuales buscan desarrollar las habilidades comunicacionales para transmitir de manera efectiva los resultados de investigación tanto a nivel oral (presentaciones y seminarios), como escrito usando medios convencionales (artículos de revista) y no convencionales (notas de política, entre otros). Esta asignatura combina un taller en estrategias de comunicación académica y la realización del "Curso Virtual de Estrategias para la publicación de la producción académica" liderado por el Centro Cultural Biblioteca Luis Echavarría Villegas. Al final de la asignatura, los estudiantes recibirán un certificado que acredita el desarrollo de la competencia. 
\item 	\emph{Seminario III}: El Seminario doctoral III se enmarca en el modelo de competencias de formación para la investigación de la Universidad EAFIT, el cual busca consolidar las competencias transversales en investigación de los estudiantes de posgrados de investigación. Este último seminario se enfoca en la consolidación de las competencias relacionadas con la gobernanza de la investigación y la especialización en la ruta de interés de los diferentes perfiles doctorales. El seminario se desarrolla a través de conversatorios con investigadores de la Universidad los cuales desde su experiencia permiten guiar la formación de los planes de carrera de los estudiantes, además, se complementará con las actividades lideradas por la Vicerrectoría de Descubrimiento y Creación. Por último, el estudiante podrá elegir un itinerario de aprendizaje de acuerdo con sus necesidades e intereses particulares en cualquiera de las siguientes rutas de formación:  empresarial, docencia e investigación para la diversidad, que buscan complementar los diferentes perfiles de investigación. Al final de la asignatura, los estudiantes recibirán un certificado que acredita el desarrollo de la competencia.
\end{itemize}

Para refrendar las competencias en el seminario II y III, los estudiantes deberán presentar el avance de su trabajo en el Seminario Doctoral un evento académico que representa el 4 hito del doctorado. De manera ideal, se espera que el estudiante durante el ciclo de investigación tenga una experiencia internacional que permita un acercamiento a la comunidad académica internacional y que contribuya a la formación de las competencias como investigador y el desarrollo de su investigación.

El hito más importante del doctorado es la defensa final de la tesis, en la cual demuestra la consolidación final de cada una de más competencias que forman el marco general propuesto por el programa.

La disertación doctoral deberá haber sido completada antes de finalizar el octavo semestre de estudios contados a partir del inicio del primer semestre de estudios. La disertación doctoral se compone de por lo menos tres artículos de alta calidad como para ser publicados en revistas nacionales o internacionales ubicadas en los cuartiles Q1 y Q2 de los índices bibliográficos ISI y/o Scopus. Por lo menos uno de los artículos debe haber sido enviados para su publicación en revistas que cumplan con los anteriores criterios. El tercer artículo debe estar terminado pero posiblemente aún requiera modificaciones menores. Por lo menos uno de los artículos debe estar escrito en inglés En todo caso, debe estar en una etapa cercana al envío para su publicación en una revista que cumpla con los criterios señalados. 

Una vez terminada el proceso académico de graduación, el estudiante iniciará el proceso administrativo para participar en las ceremonias institucionales de graduación.  Para optar al título de Doctor en Economía debe cumplir con los paz y salvos en las fechas indicadas por Admisiones y Registro para cada ceremonia de grado. Cada una de las dependencias delegadas se encargarán de verificar y confirmar su cumplimiento.