La Propuesta de Tesis buscar orientar al estudiante de doctorado en la construcción de su Disertación. La propuesta, consolida el problema de investigación que será abordado por el estudiante de doctorado, donde éste plasma los objetivos y resultados que espera alcanzar. De igual forma, define la metodología que utilizará para llegar a dichos resultados, el marco teórico y la bibliografía utilizada.  La asignatura se desarrolla a través de la asesoría presencial por parte del director del estudiante, donde el estudiante tendrá la oportunidad de interactuar con su profesor para comprender mejor el método científico, discutir casos y elaborar la propuesta de su tesis.

El curso se enmarca en el Marco de Desarrollo de Competencias -MDC- de la Escuela de Economía y Finanzas, contribuyendo a la consolidación de las siguientes áreas de dominio y descriptores:

\begin{figure}[H]
\caption{Articulación de la propuesta de tesis al MDC \label{hito1_mdc} }
\begin{center}
\includegraphics[width=\textwidth,keepaspectratio]{img/mdc_hito3.png}
\end{center}
\end{figure}

De esta manera, al final el curso lo estudiantes habrán fortalecido las siguientes competencias:
 
\section{Resultados de aprendizaje}

En el área de dominio Conocimiento y habilidades intelectuales, los resultados de aprendizaje son los siguientes:
\begin{itemize}
\item	Utiliza las principales metodologías de investigación adecuadas conforme a sus preguntas de investigación. (A1.1)
\item	Reconoce y valida problemas de investigación en economía con potencial para su investigación de acuerdo con los principios económicos aplicables. (A2.1)
\item	Construye argumentos sólidos, con premisas explícitas y conclusiones que se desprenden lógicamente de las premisas (A2.3)
\item	Analiza y evalúa críticamente la calidad de resultados de investigación y su relevancia para soportar el análisis económico en diferentes contextos. (A2.4)
\item	Diseña proyectos que retan el pensamiento tradicional en temas de investigación generales y de progreso. (A2.5)
\item	Toma una actitud creativa e imaginativa respecto a los objetos y metodologías de investigación en economía. (A3.1)
\item	Piensa de manera original, independiente y crítica sobre los principales problemas del análisis económico. (A3.2)
\end{itemize}
En el área de dominio Efectividad Personal, los resultados de aprendizaje son los siguientes:
\begin{itemize}
\item	Confía en sus propias habilidades, lo cual le permite defender exitosamente sus ideas, incursionar en nuevos desafíos en su área de investigación (B1.3)
En el área de dominio Gobernanza de la Investigación, los resultados de aprendizaje son los siguientes:
\item	Tiene conocimientos sobre el adecuado uso de los lineamientos sobre derechos de autor para la atribución de autorías y coautorías de productos de investigación. (C1.2)
\end{itemize}
En el área de dominio Compromiso e Influencia en el Medio, los resultados de aprendizaje son los siguientes:
\begin{itemize}
\item	Es capaz de comunicar efectivamente la investigación a una audiencia diversa y elabora material escrito de diversos tipos (informe, ensayo, acta) con coherencia, claridad y precisión, reconociendo la intención comunicativa y el público al que va dirigido (D2.4)
\item	Reconoce la importancia de responder por sus acciones respecto a los impactos económicos y sociales de su investigación. (D3.2)
\end{itemize}


\section{Estructura de la propuesta}

La propuesta de tesis comprende de un artículo de investigación bien desarrollado, el cual puede ser el mismo utilizado para cumplir con el requisito del artículo de segundo año, y dos ideas concretas que servirán de base para desarrollar los artículos restantes requeridos para la disertación doctoral. Es decir: 
    \begin{itemize}
        \item un proyecto/trabajo que esté bien desarrollado (no necesariamente terminado) (trabajo de segundo año);
        \item una idea razonablemente bien desarrollada para el segundo proyecto/trabajo
        \item una idea general para el tercer proyecto/trabajo.
    \end{itemize}

Si estás trabajando en una "disertación de una sola idea", entonces necesitas presentar un esquema lo suficientemente detallado de toda tu disertación para argumentar que tu proyecto es más que un solo trabajo y que serás capaz de completarlo en un tiempo razonable.


Ten en cuenta los siguientes puntos para la realización del trabajo escrito:
\begin{enumerate}
\item ¿Cuál es la principal pregunta de investigación (o preguntas, si escribe tres trabajos) que abordará la tesis?

\item ¿Por qué es importante la pregunta de investigación?: Sitúe la pregunta de investigación en la bibliografía existente y en el contexto político más amplio.

  \begin{itemize}
  \item Explique por qué es una pregunta interesante que merece más investigación.
  \item Explique qué hueco en nuestro conocimiento del tema piensa llenar.
  \end{itemize}

\item ¿Cómo piensa responder a la pregunta de investigación?
\item Explique qué modelos, datos y técnicas piensa utilizar.
  \begin{itemize}
  \item[(a)] Modelos:
  \item ¿Qué modelo o conjunto de modelos guía su enfoque?
  \item ¿Por qué estos modelos son los más adecuados para esta pregunta?

  \item[*] Si está construyendo su propio modelo, asegúrese de responder a las siguientes preguntas:
  \item ¿Qué quieres que muestre tu modelo? ¿Qué implicaciones pretende?
  \item ¿Cuáles son los supuestos clave del modelo? Sé sincero. Resalte, no oculte, los supuestos tontos.
  \item ¿En qué se diferencia su modelo de los existentes?
  \item ¿Cuáles son las ecuaciones básicas del modelo? Escríbalas. Para disertaciones con un componente empírico:


    \item[(b)] Datos:

    \item ¿Qué datos vas a utilizar para responder a la pregunta o para probar tu nuevo modelo?
    \item ¿Por qué son los datos más adecuados para responder a la pregunta?
    \item Conozca las propiedades de sus datos, por ejemplo, qué miden exactamente, cómo se recogieron, si describen a toda la población o sólo a un subconjunto.

    \item[(c)] Técnicas:

    \item ¿Qué técnicas econométricas va a utilizar?
    \item ¿Por qué son las técnicas más adecuadas?

\end{itemize}

\item ¿Cuáles son los resultados esperados o preliminares?: Presente algunos resultados preliminares, es decir, un subconjunto de los resultados que aparecerán en su disertación, o un ejemplo del TIPO de resultados que espera obtener.

\end{enumerate}

\section{Defensa Oral}

Con el fin del organizar tu presentación de una mejor manera te damos las siguientes recomendaciones:

\includegraphics[scale=0.5]{img/presentacion_propuesta.png}

Esto deja 5 minutos para preguntas y respuestas al final de su presentación. También es de esperar que le interrumpan con preguntas/comentarios/objeciones durante su presentación. (No obstante, su presentación no superará los 40-60 minutos, incluidas las interrupciones).

Ajuste la estructura anterior para que se adapte a su disertación. Por ejemplo, si aplicas los modelos económicos y econométricos de otra persona a un nuevo conjunto de datos, deberás dedicar más tiempo a presentar los detalles de tu conjunto de datos y tus resultados preliminares que a los modelos de la otra persona.

Asegúrate de que tu presentación está bien ensayada. Elimina todas las erratas de tus diapositivas y haz tu presentación ante tus amigos o frente a un espejo antes de presentarla en público.


La presentación de una propuesta de tesis doctoral es diferente a la defensa de una tesis o a un seminario. Conviene dedicar la mayor parte del tiempo de tu presentación a la descripción de lo que piensas hacer y cómo piensas hacerlo. No es necesario deslumbrar a tu audiencia con grandes resultados, guárdalos para tu defensa y para el mercado laboral.

\section{Criterios de Evaluación}

La coordinación del doctorado constituye un comité de evaluación compuesto por dos profesores, preferiblemente externos. El asesor es invitado a las discusiones, sin poder de voto sobre la aprobación/reprobación de la misma.

Tu propuesta de tesis doctoral debe demostrar que será capaz de hacer una contribución original a la disciplina y que dicha contribución es lo suficientemente importante como para ser el resultado de una tesis doctoral. Para ello, los evaluadores de su propuesta de tesis tendrán en cuenta, entre otras cosas, lo bien que puede identificar un problema adecuado para una tesis doctoral, lo bien que puede explicar que dicho problema es importante y que a otras personas les debería importar, cómo la solución de dicho problema tendrá consecuencias importantes para la disciplina o la política, cómo su contribución es original y se diferencia de los trabajos más relevantes de la literatura. Además, los árbitros evaluarán la calidad de su redacción y argumentación científica, su capacidad para seleccionar una metodología adecuada para resolver su problema y llevar a cabo su tesis con éxito

El proceso de evaluación consiste en la revisión de la propuesta de tesis escrita y la defensa pública
La propuesta de disertación doctoral deberá demostrar: 
\begin{itemize}
    \item que la investigación a desarrollar presentará una contribución original significativa al conocimiento en economía en general y al campo de énfasis en particular
    \item que la misma se desarrollará antes de finalizar el cuarto año de estudios.
\end{itemize}

El cumplimento de este y los anteriores requisitos cualifican al candidato para ser admitido como candidato al grado de Doctor en Economía.

La aprobación de la propuesta de tesis, proclama al estudiante como candidato o candidata a doctor en Economía de la Universidad EAFIT.

Las rúbricas completas de evaluación y los lineamientos para el estudiante se encuentran disponible en el Anexo

\section{Puntos Importantes}

Sólo puedes presentar tu propuesta de tesis doctoral si:
\begin{itemize}
    \item Después de haber presentado y superado los exámenes integrales,
    \item Después de haberla presentado al director del doctorado y a otros miembros del profesorado en una sección de ensayos previos,
    \item Si tu asesor de tesis y el director del doctorado están de acuerdo en que estás preparado para presentar tu propuesta de tesis doctoral.
\end{itemize}

Recuerda: 
\begin{itemize}
\item En primer lugar, discuta con su director de tesis si está preparado para presentar su propuesta de tesis doctoral.

\item Póngase en contacto con el director(a) de doctorado a cargo de la programación de la propuesta de tesis doctoral con suficiente antelación para programar su presentación. ("Con bastante antelación" significa al menos tres meses antes).
\end{itemize}

Asegúrese de que TODOS los miembros de su comité hayan leído la propuesta de tesis escrita y la hayan firmado. Existe un formulario estándar en el que deben figurar estas firmas, que se le enviará una vez finalizado con éxito su último examen exhaustivo. Si algún miembro del comité ha hecho comentarios o sugerencias sobre tu tema en el pasado, responde a esos comentarios y sugerencias.