Los principales resultados:

    \begin{itemize}
        \item Las brechas de ingreso por género son muy grandes, especialmente en los percentiles más bajos de ingreso.
        \item El ingreso de las mujeres está rezagado 10 años con respecto al de los hombres, en los percentiles más bajos.
        \item La tasa de mujeres jóvenes que ni estudian, ni trabajan es casi el doble que la de los hombres jóvenes. Esta brecha se agudizó con la pandemia.
    \item Gran brecha de ingreso entre minorías y no minorías  en todos los niveles de ingreso.
    \item A pesar de que la pobreza ha disminuido, las diferencias entre regiones todavía son muy altas.
    \item La pandemia del COVID-19 causó un aumento preocupante en las tasas de pobreza en las zonas urbanas, al punto de hacerla converger con la de las zonas  rurales.
    \end{itemize}
